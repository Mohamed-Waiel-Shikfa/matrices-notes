\documentclass{report}
\usepackage[utf8]{inputenc}
\usepackage{enumitem}
\usepackage{amsmath} % For math environments
\usepackage{tcolorbox}
\usepackage{graphicx}
\usepackage{amsmath}
\usepackage{amssymb}
\usepackage{amsthm}
\usepackage{stmaryrd}
\usepackage[short]{datetime}
\usepackage{caption}
\usepackage{subcaption}

\usepackage{multicol}
\setlength{\columnseprule}{1pt}

\usepackage{soul}
\sethlcolor{yellow}

\usepackage[margin=1.75cm]{geometry}

\theoremstyle{definition} % Non-italic style
\newtheorem{remark}{Remark}
\newtheorem{theorem}{Theorem}
\newtheorem{lemma}{Lemma}
\newtheorem{definition}{Definition}
\newtheorem{corollary}{Corollary}
\newtheorem{property}[theorem]{Property}
\newtheorem{proposition}[theorem]{Proposition}

\newcommand{\xx}{\mathbf{x}}
\newcommand{\Tx}{T(\xx)}
\newcommand{\yy}{\mathbf{y}}
\newcommand{\uu}{\mathbf{u}}
\newcommand{\ww}{\mathbf{w}}
\newcommand{\vv}{\mathbf{v}}
\newcommand{\0}{\mathbf{0}}
\newcommand{\aaa}{\mathbf{a}}
\newcommand{\bb}{\mathbf{b}}
\newcommand{\Ai}{A^{-1}}
\newcommand{\At}{A^{T}}
\newcommand{\Rn}{\mathbb{R}^n}
\newcommand{\Pn}{\mathbb{P}_n}
\newcommand{\Rm}{\mathbb{R}^m}
\newcommand{\mxn}{m \times n}
\newcommand{\Col}{\mathrm{Col}\;}
\newcommand{\Row}{\mathrm{Row}\;}
\newcommand{\Nul}{\mathrm{Nul}\;}
\newcommand{\Span}{\mathrm{Span}\;}
\newcommand{\rank}{\mathrm{rank}\;}
\newcommand{\BB}{\mathcal{B}}
\newcommand{\CC}{\mathcal{C}}
\newcommand{\cB}{[\xx]_\BB}
\newcommand{\cC}{[\xx]_\CC}
\newcommand{\pb}{P_\BB}

\newcommand{\matAI}{$\begin{bmatrix}A & I\end{bmatrix}$}


\title{21-241 Matrices and Linear Transformations Notes}
\date{Exam 2 Edition}
\author{Compiled by Salman Hajizada}

\begin{document}
\maketitle

% Example Chapter
\chapter{Linear Equations in Linear Algebra}

\section{Systems of linear equations}
\textbf{Definition:} A system of linear equations is a collection of one or more linear equations involving the same set of variables.
A system has either:
\begin{itemize}
    \item No solution
    \item Infinite solutions
    \item A unique solution
\end{itemize}
The info about a linear system can be recorded in a matrix
\[
\mathbf{A} = \begin{bmatrix}
a_{11} & a_{12} & a_{13} \\
a_{21} & a_{22} & a_{23} \\
a_{31} & a_{32} & a_{33}
\end{bmatrix}
\]
The one with just the coefficients is called \textbf{coefficient matrix}
\\The one with the last col with the product is called the \textbf{augmented matrix}
\\\\There are 3 elementary row operations to carry out on the matrix:
\begin{itemize}
    \item Replacement: by a sum of itself and a multiple of another row
    \item Interchange: swap rows
    \item Scaling: scale a row by a scalar
\end{itemize}

\section{Row Reduction and Echelon Forms}
\textbf{Definition:} A matrix is in \textbf{echelon form} if
\begin{itemize}
    \item All nonzero rows are above any rows of zeros
    \item Each leading entry of a row is in a column ro the right of the leading entry of the row above it
    \item All entries in a column below a leading entry are zeros
\end{itemize}
A matrix is in \textbf{reduced echelon form} if
\begin{itemize}
    \item It is in echelon form
    \item The leading entry in each nonzero row is 1
    \item Each leading 1 is the only nonzero entry in its column
\end{itemize}
\begin{theorem}[Uniqueness of the Reduced Echelon Form]
    Each matrix is row equivalent to one and only one reduced echelon matrix
\end{theorem}
\begin{definition}[Pivot Positon]
    A \textbf{pivot position} in A is a location in A that corresponds to a leading 1
    in the reduced echelon form of A
\end{definition}
\begin{tcolorbox}[colback=blue!5!white, colframe=blue!75!black, title=Row Reduction Algorithm]
\begin{enumerate}
    \item Start with left non-zero column. This is a pivot column. Pivot position is at the top
    \item Choose a nonzero entry in the pivot column as a pivot. Interchange to move it into pivot position
    \item Use row replacement to make zeros below pivot position
    \item Ignore the row with the pivot and the ones above. Apply steps 1-3 to the remaining matrix until no more rows to modify
    \item Starting with rightmost pivot and working up and left, make zeros above each pivot. Make every pivot a 1 by scaling
\end{enumerate}
\end{tcolorbox}
Variables corresponding to pivot columns are \textbf{basic}, the other ones are \textbf{free}
\begin{theorem}[Existence and Uniqueness theorem]
\end{theorem}

A linear system is consistent iff the right column of the augmented matrix is NOT a pivot column
i.e. iff echelon form has NO row of form:
\[\begin{bmatrix}
    0 & \ldots & 0 & b
\end{bmatrix} \text{with $b$ nonzero}
\]
If a system is consistent then solution set contain either a unique solution if there are 0 free variables
or infinite solutions if there is at least 1 free variable.

\section{Vector Equations}
\begin{tcolorbox}[colback=blue!5!white, colframe=blue!75!black, title=Algebraic Properties of Vectors in $\mathbb{R}^n$]
$\forall$ $\mathbf{u}, \mathbf{v}, \mathbf{w} \in \mathbb{R}^n$ and scalars $c$ and $d$:
\begin{enumerate}
    \item \textbf{Commutativity of addition:} $\mathbf{u} + \mathbf{v} = \mathbf{v} + \mathbf{u}$
    \item \textbf{Associativity of addition:} $\mathbf{u} + (\mathbf{v} + \mathbf{w}) = (\mathbf{u} + \mathbf{v}) + \mathbf{w}$
    \item \textbf{Additive identity:} $\mathbf{u} + \mathbf{0} = \mathbf{u}$
    \item \textbf{Additive inverse:} $\mathbf{u} + (-\mathbf{u}) = \mathbf{0}$
    \item \textbf{Distributivity of scalar multiplication over addition (vectors):} $c(\mathbf{u} + \mathbf{v}) = c\mathbf{u} + c\mathbf{v}$
    \item \textbf{Distributivity of scalar multiplication over addition (scalars):} $(c + d)\mathbf{u} = c\mathbf{u} + d\mathbf{u}$
    \item \textbf{Associativity of scalar multiplication:} $(cd)\mathbf{u} = c(d\mathbf{u})$
    \item \textbf{Scalar identity:} $1 \mathbf{u} = \mathbf{u}$
\end{enumerate}
\end{tcolorbox}
    
A linear combination is a vector $\mathbf{y} = c_1\mathbf{v}_1 + \ldots + c_p\mathbf{v}_p$

\begin{definition}
    $Span\{\mathbf{v}_1, \ldots, \mathbf{v}_p\}$ is the set of all linear combinations of $\mathbf{v_1}, \ldots, \mathbf{v_p}$
    \\It is the set of all vectors that can be written in the form $c_1\mathbf{v}_1 + \ldots + c_p\mathbf{v}_p$
\end{definition}

\section{The Matrix Equation Ax = b}
\begin{definition}
\end{definition}
If A is an $m \times n$ matrix, with columns $\mathbf{a_1}, \ldots, \mathbf{a_n}$ and if $\mathbf{x} \in \mathbb{R}^n$,

\[
A \mathbf{x} = 
\begin{bmatrix}
\mathbf{a}_1 & \mathbf{a}_2 & \cdots & \mathbf{a}_n
\end{bmatrix}
\begin{bmatrix}
x_1 \\ x_2 \\ \vdots \\ x_n
\end{bmatrix}
= x_1 \mathbf{a}_1 + x_2 \mathbf{a}_2 + \cdots + x_n \mathbf{a}_n
\]

This is defined only if the number of columns of A equals number of entires in $\mathbf{x}$

\begin{theorem}
\end{theorem}
If A is an $m \times n$ matrix and $\mathbf{b} \in \mathbb{R}^n$, then $A \mathbf{x} = \mathbf{b}$ has the same solution set as 
\[
x_1 \mathbf{a}_1 + x_2 \mathbf{a}_2 + \cdots + x_n \mathbf{a}_n = b
\]
which has the same solution set as the system of linear equations with matrix
\[
\begin{bmatrix}
    \mathbf{a}_1 & \ldots & \mathbf{a}_n & \mathbf{b}
\end{bmatrix}
\]

\begin{theorem}
\end{theorem}
Let A be a $m \times n$ matrix. The following statements are equivalent
\begin{enumerate}[label=\alph*.]
    \item $\forall\ \mathbf{b} \in \mathbb{R}^m, A \mathbf{x} = \mathbf{b}$ has a solution
    \item $\forall\ \mathbf{b}$ is a linear combination of the columns of A
    \item The columns of A span $\mathbb{R}^m$
    \item A has a pivot in every row
\end{enumerate}
Warning: A is a \textbf{coefficient matrix} in this theorem

\begin{theorem}
\end{theorem}
If A is an $m \times n$ matrix and $\mathbf{u},\mathbf{v} \in \mathbb{R}^n$, and c is a scalar
\begin{enumerate}[label=\alph*.]
    \item $A\ (\mathbf{u} + \mathbf{v}) = A \mathbf{u} + A \mathbf{v}$;
    \item $A\ (c \mathbf{u}) = c (A \mathbf{u})$;
\end{enumerate}

\section{Solution sets of Linear Systems}

\textbf{Homogenous Linear Systems}: if it can be written in form $A \mathbf{x} = \mathbf{0}$
\\This system always has at least one solution ($\mathbf{x} = \mathbf{0}$): \textbf{trivial solution}
\\The equation has a \textbf{non-trivial} solution iff it has at least one free variable
\\\\\textbf{Nonhomogenous Linear Systems}: if it can be written in form $A \mathbf{x} = \mathbf{b}$, 
$\mathbf{b}$ non-zero

\begin{theorem}
\end{theorem}
Let $A \mathbf{x} = \mathbf{b}$ be consistent for some $\mathbf{b}$, let $\mathbf{p}$ be a solution.
Then the solution set is the set of all vectors in form $\mathbf{w} = \mathbf{p} + \mathbf{v}_h$, 
where $\mathbf{v}_h$ is any solution of $A \mathbf{x} = \mathbf{0}$

\begin{tcolorbox}[colback=blue!5!white, colframe=blue!75!black, title=Writing a solution set in parametric vector form]
\begin{enumerate}
    \item Row reduce to reduced echelon form
    \item Express each basic variable in terms of any free variables
    \item Write a typical solution as a vector whose entries depend on free variables
    \item Decompose it into a linear combination of vectors (with numbers only inside) using free variables as parameters
\end{enumerate}
\end{tcolorbox}

\section{Application of Linear Systems} idk how to make it skip it. So just ignore this
\section{Linear Independence}
\begin{definition}
\end{definition}
$\{ \mathbf{v}_1, \ldots, \mathbf{v}_p\}$ is \textbf{linearly independent} if 
\[
x_1 \mathbf{v}_1 + \ldots + x_p \mathbf{v}_p = \mathbf{0}
\]
has only the trivial solution. It is said to be \textbf{linearly dependent} if $\exists c_1, \ldots, c_p$, not all zero, such that 
\[
c_1 \mathbf{v}_1 + \ldots + c_p \mathbf{v}_p = \mathbf{0}
\]

A set of two vectors is linearly dependent if at least one is a multiple of the other

\begin{theorem}[Characterization of Linearly independent Sets]
\end{theorem}
A set of two or more vectors is linearly dependent iff at least one of them is a linear combination the others.

\begin{theorem}
\end{theorem}
If a set contains more vectors than entries in each vector, then it is linearly dependant 

\begin{theorem}
\end{theorem}
If a set contains the zero vector, then it is linearly dependant 

\section{Introduction to Linear Transformations}

A \textbf{transformation} $T$ from $\Rn$ to $\Rm$ is a rule that assigns each vector $\xx$ in 
$\Rn$ to a vector $\Tx$ in $\Rm$
\begin{itemize}
    \item Domain: $\Rn$
    \item Codomain: $\Rm$
    \item Range: set of all images $\Tx$
\end{itemize}

\begin{definition}
    $T$ is linear iff \begin{itemize}
        \item $T(\uu + \vv) = T(\uu) + T(\vv)$
        \item $T(c\uu) = c T(\uu)$
    \end{itemize}
\end{definition}

If $T$ is linear then:
\begin{itemize}
    \item $T(\0) = \0$
    \item $T(c\uu + d\vv) = cT(\uu) + dT(\vv)$
    \item $T(c_1\vv_1 + \ldots + c_p\vv_p) = T(c_1\vv_1) + \ldots + T(c_p\vv_p)$
\end{itemize}

\section{The Matrix of a Linear Transformation}

\begin{theorem}
\end{theorem}
For T, there exists a unique A such that
\[
T(\mathbf{x}) = A \mathbf{x}\ \ \forall \mathbf{x} \in \mathbb{R}^n\\
\]
Where $A = \begin{bmatrix} T(\mathbf{e}_1) & T(\mathbf{e}_2) & \dots & T(\mathbf{e}_n) \end{bmatrix}$\\
With $\mathbf{e}_j$ is the $j$th column of the identity matrix


\begin{definition}
\end{definition}
$T$ is \textbf{onto} (surjective) if
\begin{itemize}
    \item Range of $T$ = Codomain
    \item $\forall \ \mathbf{v} \in codomain (B), \exists \ 
    \mathbf{u} \in domain (A)$ such that $T(\mathbf{u})=\mathbf{v}$
    \item Find standard matrix 
    \[
    A = \begin{bmatrix}
        T(c_1) & \ldots & T(c_n)
    \end{bmatrix}
    \]
    Onto if $A \mathbf{u} = \mathbf{v}$ has a solution for each $\mathbf{v}$ in range
\end{itemize}

\begin{definition}
\end{definition}
$T$ is \textbf{one-to-one} (injective)if
\begin{itemize}
    \item $T(\mathbf{u}_1) = T(\mathbf{u}_2) \implies \mathbf{u}_1 = \mathbf{u}_2$
    \item Find standard matrix 
    \[
    A = \begin{bmatrix}
        T(c_1) & \ldots & T(c_n)
    \end{bmatrix}
    \]
    $A \mathbf{u} = \mathbf{b}$ has a unique solution for each $\mathbf{b} \in Range(T)$
\end{itemize}

\begin{theorem}
    $T$ is one-to-one iff $T(\mathbf{x}) = \mathbf{0}$ has a ONLY the trivial solution
\end{theorem}

\begin{theorem}
    Let $T: \Rn \to \Rm$, let $A$ be standard matrix for $T$. Then 
    \begin{itemize}
        \item $T$ maps $\Rn$ to $\Rm$ iff the columns of $A$ span $\Rm$
        \item $T$ is one-to-one iff the columns of $A$ are linearly independent
    \end{itemize}
\end{theorem}

\chapter{Matrix Algebra}

\section{Matrix Operations}
\begin{itemize}
    \item \textbf{Square Matrix}: row and columns are the same
    \item \textbf{Diagonal entries}: $a_{11}, a_{22}, \ldots$
    \item \textbf{Diagonal matrix}: square $n \times n$ matrix who's non-diagonal entires are zero
    \item \textbf{Zero matrix}: $m \times n$ matrix where all entries are 0
\end{itemize}

$\left[ a_{ij} \right]_{m \times n} \pm \left[ b_{ij} \right]_{m \times n} = \left[ a_{ij} \pm b_{ij} \right]_{m \times n}$
\\\\
Two matrices are \textbf{equal} if they have the same size and their entries are equal (duh)

\setcounter{theorem}{0}
\setcounter{definition}{0}
\begin{theorem}
\end{theorem}
\begin{tcolorbox}[colback=blue!5!white, colframe=blue!75!black, title=Properties of Matrices]
Let $A$, $B$ and $C$ be matrices of the same size, and let $r$ and $s$ be scalars
\begin{itemize}
    \item $A + B = B + A$
    \item $(A + B) + C = A + (B + C)$
    \item $A + 0 = A$
    \item $r(A + B) = rA + rB$
    \item $(r + s)A = rA + sA$
    \item $r(sA) = (rs)A$
\end{itemize}
\end{tcolorbox}

\begin{definition}
    If $A$ is a $m \times n$ matrix and $B$ is an $n \times p$ matrix with columns
$\mathbf{b}_1, \ldots, \mathbf{b}_p$, then 
\[
AB = A\begin{bmatrix}\mathbf{b}_1 \ldots \mathbf{b}_p\end{bmatrix}
    = \begin{bmatrix}A\mathbf{b}_1 \ldots A\mathbf{b}_p\end{bmatrix}
\]
\end{definition}

\begin{itemize}
\item Each column of $AB$ is a linear combination of the columns of A using weights from columns of B
\item \textbf{The number of columns of $A$ must match the number of rows in $B$ }
\item Row column rule: $(AB)_{ij} = a_{i1}b_{1j} + a_{i2}b_{2j} + \ldots + a_{in}b_{nj}$
\item $row_i(AB) = row_i(A)\cdot B$
\end{itemize}

\begin{theorem}
\end{theorem}
\begin{tcolorbox}[colback=blue!5!white, colframe=blue!75!black, title=Properties of Matrices]
Let $A$ be an $m \times n$ matrix, and let $B$ and $C$ have sizes for which the indicated sums and products are defined.

\begin{itemize}
    \item[] $A(BC) = (AB)C$ \hfill (associative law of multiplication)
    \item[] $A(B + C) = AB + AC$ \hfill (left distributive law)
    \item[] $(B + C)A = BA + CA$ \hfill (right distributive law)
    \item[] $r(AB) = (rA)B = A(rB)$ for any scalar $r$
    \item[] $I_m A = A = A I_n$ \hfill (identity for matrix multiplication)
\end{itemize}
\end{tcolorbox}

\begin{tcolorbox}[colback=blue!5!white, colframe=red!75!black, title=WARNING]
Let $A$ be an $m \times n$ matrix, and let $B$ and $C$ have sizes for which the indicated sums and products are defined.

\begin{itemize}
    \item In general, $AB \ne BA$
    \item In general, $AB = AC$ \textbf{does not imply} $B = C$
    \item In general, $AB = 0$ \textbf{does not imply} $A = 0 \vee B = 0$
\end{itemize}
\end{tcolorbox}

Power of a matrix: $A^k = \underbrace{A A \dots A}_{k}$ \\

The transpose of a matrix: \( A^T \) is obtained by swapping its rows and columns of A.
\begin{theorem}\end{theorem}

\begin{tcolorbox}[colback=blue!5!white, colframe=blue!75!black]
\begin{itemize}
    \item $(A^T)^T = A$ 
    \item $(A + B)^T = \At + B^T$ 
    \item $(rA)^T = r A^T$ 
    \item $(AB)^T = B^T A^T$ 
\end{itemize}
\end{tcolorbox}

\section{The Inverse of a Matrix}

An \( n \times n \) matrix \( A \) is said to be \textbf{invertible} if there is an \( n \times n \) matrix \( C \) such that
\[
C A = I_n \quad \text{and} \quad A C = I_n,
\]
In this case, $C$ is an \textbf{inverse} of $A$
\[
A^{-1} A = I \text{ and } A A^{-1} = I
\]

\begin{theorem}
    Let $A = \begin{bmatrix}
        a & b \\ c & d
    \end{bmatrix}$ 
    If $ad - bc \ne 0$, then A is invertEBIL and 
    \[
    A^-1 = \frac{1}{ad -bc}\begin{bmatrix}
        d & -b \\ -c & a
    \end{bmatrix}
    \]
\end{theorem}

$det A = ad - bc$

\begin{theorem}
    If A is invertible, then $\forall \ \mathbf{b} \in \mathbb{R}^n$, 
    $A\mathbf{x} = \mathbf{b}$  
    has the unique solution $\mathbf{x} = A^{-1}\mathbf{b}$
\end{theorem}
\vspace{1cm}
\begin{theorem}
\begin{enumerate}
    \item If A invertible, then $\Ai$ too and $(\Ai)^{-1} = A$
    \item If A and B invertible, $(AB)^{-1} = B^{-1}\Ai$
    \item If A invertible, $(A^T)^{-1} = (\Ai)^T$
\end{enumerate}
\end{theorem}

\begin{itemize}
    \item An \textbf{elementary matrix} is a matrix obtained by performing a single 
elementary row operation on $I$
    \item If a row op if performed on A, the result can be written as EA, where E is made
    by performing the same row op on I
    \item Each E is an invertible and its inverse is the elementary matrix that transforms E back into I. 
\end{itemize}

\begin{theorem}
    A is invertible iff A is row equivalent to I, and any sequence of row ops that reduces A 
    to I also transforms I to $\Ai$
\end{theorem}

\textbf{Finding }$\Ai$: Row reduce \matAI. If A row-eq to I then \matAI is row-eq to
$\begin{bmatrix}I & \Ai\end{bmatrix}$

\setcounter{section}{2}
\section{Characterizations of Matrices}

\setcounter{theorem}{7}
\begin{theorem}\end{theorem}
\begin{tcolorbox}[colback=blue!5!white, colframe=blue!75!black]
The following statements about a square n by n matrix A are equivalent
\begin{itemize}
    \item \( A \) is an invertible matrix.
    \item \( A \) is row equivalent to the \( n \times n \) identity matrix.
    \item \( A \) has \( n \) pivot positions.
    \item The equation \( Ax = 0 \) has only the trivial solution.
    \item The columns of \( A \) form a linearly independent set.
    \item The linear transformation \( x \mapsto Ax \) is one-to-one.
    \item The equation \( Ax = b \) has at least one solution for each \( b \in \mathbb{R}^n \).
    \item The columns of \( A \) span \( \mathbb{R}^n \).
    \item The linear transformation \( x \mapsto Ax \) maps \( \mathbb{R}^n \) onto \( \mathbb{R}^n \).
    \item There is an \( n \times n \) matrix \( C \) such that \( CA = I \).
    \item There is an \( n \times n \) matrix \( D \) such that \( AD = I \).
    \item \( A^T \) is an invertible matrix.
\end{itemize}
\end{tcolorbox}

If $AB = I$, then $A$ and $B$ invertible with $B = \Ai$ and $A = B^{-1}$

A linear transformation is invertible if $\exists S: \Rn \to \Rn$ such that:
\[
\begin{array}{rlr}
    S(T(\xx)) = x & \text{for all } \xx \in \Rn \\
    T(S(\xx)) = x & \text{for all } \xx \in \Rn \\
\end{array}
\]

\begin{theorem}
    Let $T : \Rn \to \Rn$ and $A$ be the matrix. T is invertible iff A is invertible.
    Then $S(\xx) = \Ai \xx $
\end{theorem}

\section{Partitioned Matrices}

Trivial stuff honestly just partition and multiply.

Ok actually there is one good theorem here
\begin{theorem}[Column-Row Expansion of AB]\end{theorem}
If A is $m \times n$ and B is $n \times p$:
\[
AB = \begin{bmatrix}col_1(A) & \ldots & col_n(A)\end{bmatrix} \begin{bmatrix}row_1(B) \\ \vdots \\ row_n(B)\end{bmatrix} = col_1(A) row_1(B) + \ldots + col_n(A) row_n(B) 
\]

The inverse of a block upper triangular matrix:
\[
\Ai = \begin{bmatrix}
    A_{11} & A_{12} \\
    0 & A_{22}
\end{bmatrix} = \begin{bmatrix}
    A_{11}^{-1} & -A_{11}^{-1}A_{12}A_{22}^{-1} \\
    0 & A_{22}^{-1}
\end{bmatrix}
\]

\section{Matrix Factorizations}

Assume $A$ is $m \times n$ and can be reduced to Echelon Form without row interchanges. \\
Then $A = LU$, 
\begin{itemize}
    \item $L: m \times m$ lower triangular with 1's in diagonal
    \item $U: m \times n$ echelon form of $A$
\end{itemize} 

$A\xx = \bb$ can be written as $L(U\xx) = \bb$. Writing $\yy = U\xx$, we get:
\[
\begin{array}{rlr}
    L(U\xx) = \bb \\\\
    L\yy = \bb \\
    U\xx = \yy
\end{array}
\]
We can get $\xx$ by solving the easy pair of equations:
\begin{enumerate}
    \item Solve $L\yy = \bb$ for $\yy$
    \item Solve $U\xx = \yy$ for $\xx$
\end{enumerate}

\begin{tcolorbox}[colback=blue!5!white, colframe=blue!75!black, title=LU Factorization Algorithm (boring)]
\begin{enumerate}
    \item Reduce A to echelon form $U$ using a sequence of row replacement operations
    \item Place entries in $L$ such that the same sequence of row operations reduces L to I
\end{enumerate}
\end{tcolorbox}

\begin{tcolorbox}[colback=blue!5!white, colframe=yellow!75!black, title=LU Factorization Algorithm (fast)]
For a matrix $A: m \times n$
\begin{enumerate}
    \item Initialize $L$ as the identity matrix $m \times m$ and $U$ as $A$.
    \item For each row $i = 1, 2, \dots, m$:
    \begin{itemize}
        \item Use row operations to create zeros below the diagonal in $U$ by subtracting multiples of the current row from the rows below.
        \item For each row operation, divide the corresponding entries below the pivot in column $i$ by the pivot, and store these multipliers in the corresponding positions of $L$.
    \end{itemize}
    \item Continue this process until $U$ is in echelon form and $L$ has recorded the multipliers such that $A = LU$.
\end{enumerate}
\end{tcolorbox}
This description is very bad, please practice this on a non-square matrix 

\chapter{Determinants}
\section{Intro to Determinants}
\setcounter{theorem}{0}
\setcounter{definition}{0}

First, we denote $A_{ij}$ to be the sub-matrix of $A$ formed by deleting the $i$th row and $j$th column of A

\begin{definition}
For $n \ge 2$, the \textbf{determinant} of a $n \times n$: \[
\begin{array}{rlr}
    \det A &= a_{11} \det(A_{11}) - a_{12} \det(A_{12}) + \ldots + (-1)^{1+n} a_{1n} \det(A_{1n}) = \\    
    &= \sum_{j = 1}^{n}(-1)^{1+j}a_{1j}\det(A_{1j})
\end{array}
\]
\end{definition}

\textbf{The (i, j)-cofactor of A}: $\cof = (-1)^{i+j}\det(A_{ij})$

\begin{theorem}
    The determinant of A can be computed by cofactor expansion across any row or column.\\
    Across row i:
    \[
    \det(A) = a_{i1}C_{i1} + \ldots + a_{in}C_{in}
    \]
    Across column j:
    \[
    \det(A) = a_{1j}C_{1j} + \ldots + a_{nj}C_{nj}
    \]
\end{theorem}

Signs of the cofactor:
\[
\begin{bmatrix}
    + & - & + & \ldots \\
    - & + & - & \\
    + & - & + & \\
    \vdots & & & \ddots
\end{bmatrix}
\]

\begin{theorem}
    If A triangular, then $\det(A)$ is the product of the entires on the main diagonal of $A$
\end{theorem}

\section{Properties of Determinants}

\begin{theorem}\end{theorem}
\begin{tcolorbox}[colback=blue!5!white, colframe=blue!75!black, title=Row Operations]
For a matrix $A: m \times n$
\begin{enumerate}[label=\alph*.]
    \item If a multiple of a row in $A$ is added to another to produce $B$: $\det(B) = \det(A)$
    \item If two rows of $A$ are swapped to produce $B$: $\det(B) = -\det(A)$
    \item If a row of $A$ is multiplied by $k$ to produce $B$: $\det(B) = k \cdot \det(A)$
\end{enumerate}
\end{tcolorbox}
Two main strats where this is useful:
\begin{itemize}
    \item Row reduce a matrix to echelon form to simplify calculation (without row interchange!)
    \item Factor out a common factor from a row to simplify calculation 
\end{itemize}

\begin{theorem}
    $A$ is invertible iff $\det(A) \ne 0$ (trivial)
\end{theorem}

\begin{theorem}
    $\det(A) = \det(\At)$
\end{theorem}

\begin{theorem}
    If $A$ and $B$ $n \times n$, then $\det(AB) = \det(A) \cdot \det(B)$
\end{theorem}

\begin{lemma}
    $\det(\Ai) = \frac{1}{\det(A)}$
\end{lemma}

\section{Crammers Rule}
Let $A_i(\bb)$ be the matrix obtained from $A$ by replacing column $i$ by the vector $\bb$

\begin{theorem}[Cramer's Rule]
    Let $A: n \times n$, invertible, the solution $\xx$ of $A\xx = \bb$ is given by: \[
    x_i = \frac{\det(A_i(\bb))}{\det(A)}, \text{   for }i = 1, 2, \ldots, n
    \]
\end{theorem}

The matrix of cofactors is called the \textbf{adjugate or classical adjoint} of $A$,
denoted by adj $A$

\begin{theorem}[an Inverse Forumla]
    \[
    \Ai = \frac{1}{\det(A)}\begin{bmatrix}
        C_{11} & C_{21} &\ldots & C_{n1} \\
        C_{12} & C_{22} &\ldots & C_{n2} \\
        \vdots & \vdots & \\
        C_{1n} & C_{2n} &\ldots & C_{nn} \\
    \end{bmatrix} = \frac{adj(A)}{\det(A)}
    \] 
\end{theorem}

\begin{theorem}
    If $A: 2 \times 2$, Area of parallelogram determined by columns of A is $|\det(A)$\\
    If $A: 3 \times 3$, Volume of parallelepiped determined by columns of A is $|\det(A)$\\
\end{theorem}

\chapter{Vector Spaces and Subspaces}

\section{Vector Spaces and Subspaces}

\begin{definition}
    A \textbf{vector space} is a nonempty set $V$ of objects called vectors, on which there are defined two operations:
    \textit{addition} and \textit{scalar multiplication}. The axioms below must hold for all $\uu, \ww, \vv \ \in V$
\end{definition}
\begin{tcolorbox}[colback=blue!5!white, colframe=blue!75!black]
\begin{enumerate}
    \item $\uu + \vv \ \in V$ (Closure under addition)
    \item $\uu + \vv = \vv + \uu$
    \item $(\uu + \vv) + \ww = \uu + (\vv + \ww)$
    \item $\0 \ \in V$, such that $\uu + \0 = \uu$
    \item $\forall \ \uu \ \in V, \ \exists \ -\uu \ \in V$ such that $\uu + (-\uu) = \0$
    \item $c \uu \ \in V$
    \item $c (\uu + \vv) = c \uu + c \vv$
    \item $(c + d) \uu = c \uu + d \uu$
    \item $c (d \uu) = (c d) \uu$
    \item $1 \uu = \uu$
\end{enumerate}
\end{tcolorbox}

\begin{definition}
    A \textbf{subspace} of a space $V$ is a subset $H$ of $V$ that:
    \begin{itemize}
        \item The zero vector of $V$ is in $H$
        \item $H$ is closed under vector addition: $\uu + \vv \ \in H$ for all $\uu, \vv \ \in H$
        \item $H$ is closed under scalar multiplication: $c\uu \ \in H$ for all scalars $c$ and vectors $\uu\ \in H$
    \end{itemize}
\end{definition}

\begin{theorem}
    If $\vv_1, \ldots, \vv_p$ are in a vector space $V$, then $Span \{\vv_1, \ldots, \vv_p\}$ is a subspace of $V$
\end{theorem}
\end{document}